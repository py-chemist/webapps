
\documentclass{minimal}
\usepackage{xcolor, chemfig, mol2chemfig}
\usepackage[paperheight=6cm, paperwidth=28cm]{geometry}
\usepackage{amsmath}
\setatomsep{2em}
\setbondoffset{1pt}
\setdoublesep{3pt}
\setbondstyle{line width=1pt}

\begin{document}
\vspace*{\fill}
\vspace{-4pt}
\begin{center}
\schemestart[0,2.5,thick]
\chemfig{
            % 1
     -[:330]% 2
    =_[:270]% 3
     -[:210]% 4
    =_[:150]% 5
      -[:90]% 6
               (
         =_[:30]% -> 1
               )
}
\arrow{->[CH$_3$C$_6$H$_5$][110$^\circ$C]}
\chemfig{
           % 1
    -[:330]% 2
    -[:270]% 3
    -[:210]% 4
    -[:150]% 5
     -[:90]% 6
              (
         -[:30]% -> 1
              )
}
\arrow{->[toluene][reflux]}
\chemfig{
            % 1
     -[:330]% 2
    =_[:270]% 3
     -[:210]% 4
    =_[:150]% 5
      -[:90]% 6
               (
         =_[:30]% -> 1
               )
}
\+{1em, 1em, -2em}
\chemfig{
           O% 8
    =[:270]% 7
              (
        -[:330]% 9
         =[:30]% 10
              )
    -[:210]% 2
    -[:150]% 1
    -[:210]% 6
    -[:270]% 5
    -[:330]% 4
     -[:30]% 3
              (
         -[:90]% -> 2
              )
}
\schemestop
\end{center}
\vspace*{\fill}
\end{document}